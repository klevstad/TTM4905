\chapter{Background}
\label{chp:background}

\section{Internet of Things}

% Bruke gls på iot?
Over the last decade, a concept called the \emph{Internet of Things} has gained increased attention, both from the research community, as well as commercial actors and consumers. The term \gls{iot} was, accordingly to most sources, coined in 1999 by the British visionary Kevin Ashton in a presentation about \gls{rfid} \cite{iot-phrase-2} \cite{iot-phrase-1}. Ashton's definition of the concept was a world where computers do not depend on human beings to provide them with information. Out of all the petabytes of information available on the Internet, most of it has been created and captured by humans performing some sort of action. In his opinion, \gls{iot} is more about providing computers with the ability to gather information on their own.


% Bedre dette

%Another vision of the \gls{iot} is based upon the mixing of captured sensor data and connection of physical things to the Internet. 

A computational device containing some sort of sensor is attached to your everyday physical device, creating a bridge between our physical world and the cyber world \cite{Kopetz2011}. The connection to the Internet allows us to monitor and control these devices and sensors from a remote distance. Another vital part of \gls{iot} is device-to-device communications, essentially enabling devices to communicate with each other without human aid, and exchange and retrieve information.


% Where are we now

The next step of \gls{iot} is device-to-device communication, efficiently enabling devices to retrieve and exchange information. Such devices could be sensors monitoring some operation, a physical area, or even devices attached to a physical body. The possibilities are more or less unlimited. Imagine a home automation and surveillance system for your cabin, where both lights, heaters, smoke detectors, underfloor heating, motion detectors, security cameras, garage and so on, are all interconnected with each other through small wireless devices.  As it is called the \emph{internet} of things for a reason, your system and devices would be accessible over the Internet, allowing you to monitor the current status of your cabin remotely from your couch at home, as well as looking at historical data of the different sensors and devices. When the weekend arrives and you head for the mountains, the \gls{iot} provides you with an opportunity to preheat different (or all) sections of the cabin, deactivate the alarm, and perhaps instruct the sauna to start getting cosy. 

Another approach is to avoid using a screen to remotely control the system, and instead allowing the system to observe and act on your behaviour. We want the devices to know us and figure out the correct thing to do without us telling them. For example, when pulling your car in the driveway, you want the garage door that is connected with your car to open up. The garage notifies your front door that you are home, which conveniently unlocks and notifies your house to turn on the lights in your hallway and perhaps the heater in your living room.


% Where is it going


The possibilities that are revealed as the \gls{iot} grows larger and the services expands are infinite. The concept is highly applicable for different scenarios involving home automation, standalone consumer products, industrial and environmental facilities, as well as medical surveillance. While larger automation systems for homes and facilities have been the target for the research community and early adopters, the consumer market has been focused on so-called \emph{wearables}. Wearables are fundamentally devices that you wear, such as smart watches, fitness trackers, virtual reality glasses, headphones and smart clothing. Such human-centric devices is less about automation, and more focused on personal improvement. Nevertheless, the increase in \gls{iot} devices possibly provides us with a more cost efficient future, both in our use of time, as well as energy and consumption of other resources.


% Security challenges
As the \gls{iot} is built upon the Internet, it faces the same types of security issues as the Internet itself. The amount of "things" connected to the Internet is calculated to be 6.4 billions by the end of 2016, which is almost a 30\% increase from 2015. By 2020, the expected number of these "things" is more than 20 billion \cite{iot-gartner}, providing attackers with equally many possible devices to attack. Given the knowledge that some of these devices may be medical (and other sensitive applications), we quickly recognize potential catastrophic scenarios.


The \gls{iot} architecture is divided into three layers; perception, transportation and application, each providing different types of security \cite{Jing2014}. Transportation and application layers, however, are out of scope for this project. Perception layer is mostly about sensors and other nodes that collects information from its environment, and communicates it throughout the transport network. Another object for the layer is to pass control messages received. Examples of such technologies are \gls{rfid}, \gls{wsn} and \gls{gps}, each dealing with their own security problems and solutions. The most adjacent problems to the scope of this thesis are the problems related to key establishment and key management, which defines how two devices safely can establish secure communication between each other.

In an \gls{iot} world, the protection of data and privacy is an essential part. As previously mentioned, \gls{iot} technology may be a solution for problems involving sensitive information. In a medical facility, a possible scenario could be a \gls{wsn}, which are dynamic and bi-directional network of nodes where each node has one or more sensors connected to it. A patient may have sensors implanted in its body, as well as different instruments attached for measuring different properties. All these devices communicate with each other wirelessly, and the network is therefore a possible target for an attacker. To prevent the attacker from eavesdropping, and possible forging content in the network, encryption and authentication of the different nodes are crucial.



More motivation for the need for key management/establishment in \gls{iot}?



\section{The IEEE 802.15.4 Standard}

Following the evolution of \gls{iot}, the need for cheap devices to communicate efficiently between each other has arose. Existing architectures such as 802.11 (WiFi) and Bluetooth are too expensive in terms of processing and energy consumption, as the idea of \gls{iot} is to connect even the smallest devices to a network or Internet. As these devices are small, they have a limited amount of battery, and hence need to use it in a highly efficient matter.

\begin{figure}[h]
	\centering
	\includegraphics[scale=0.65]{osi.png}
	\caption{The \gls{osi} stack with layers, the data they carry, and example of technology running on the different layers.}
	\label{fig:osi}
\end{figure}

802.15.4 protocols are envisioned for applications supporting smart homes, medical surveillance, monitoring systems for environmental and industrial systems, as well as sensor systems for heating and ventilation. As we know from the \gls{iot}, it is really the imagination that puts an end to the possibilities for interconnected devices. The \gls{osi} stack defines the internal structure of communications systems, and is shown in Figure \ref{fig:osi}.

As the 802.15.4 standard only defines the physical and data link (\gls{mac}) layer of the \gls{osi} stack, specifications need to be developed to utilize the possibilities provided by 802.15.4 in the upper layers. ZigBee, maintained by the ZigBee Alliance, is the most notable example of specifications that uses 802.15.4 as its base. Others include WirelessHART, MiWi, and ISA100.11a. Cite these? Homepage?

% cite

The fundamental intention of the 802.15.4 standard is to provide low-rate, low-power communication between devices within a sensor network or \gls{wpan}. Its main use case is to let multiple devices within a short range communicate with each other over a low-rate radio, while maintaining a modest energy consumption. Figure \ref{fig:802154-figure} paints a picture of what 802.15.4 is compared to more well-established concepts such as WiFi (802.11) and Bluetooth, focusing on energy consumption, complexity and date rate. While being a smaller, more cost efficient device than those found in more complex networks, 802.15.4 networks have a much more limited range (about 10 meters), and in most cases a throughput below 250 Kbps \cite{gutierrez2001ieee}. Not only is the 802.15.4 standard significantly lighter in terms of data rate and power consumption, it is also aimed on a different market than regular \gls{wpan}s. \gls{wpan}s are oriented around a person, creating a personal network for the user, which has higher demands to data rate, and can allow a higher energy consumption. 802.15.4, however, focuses on interconnecting devices that do not necessary have this constraint, such as industrial and medical applications. 


\begin{figure}[h]
	\centering
	\includegraphics[scale=0.45]{802154.png}
	\caption{Figure of IEEE 802.15.4's operational space compared to other wireless standards \cite{gutierrez2001ieee}.}
	\label{fig:802154-figure}
\end{figure}

Four basic security services are provided in the 802.15.4 link-layer security package, namely access control, message integrity, message confidentiality and replay protection (sequential freshness) \cite{sastry2004security}. 802.15.4 is delivered with a total of eight different security suites, providing none, some or all of the described security services, and it is up to the application designer to specify and enable the desired security properties. In the most secure end of the scale we find \gls{aes}-\gls{ccm}, which is encryption using the block cipher \gls{aes} with either 32, 64 or 128-bit \gls{mac-auth}. Such a suite provides both strong encryption and possibly unforgeable messages (a 64-bit \gls{mac-auth} gives an adversary a $2^{-64}$ chance of successfully forging a message, and is used to enable legitimate nodes in the network to detect if a received message have been tampered with). In the other end of the scale we find a suite providing only confidentiality using \gls{aes} in \gls{ctr} mode. This suite does, however, not provide any form of authentication giving adversaries the possibility to forge messages, which can not be said to be especially secure.


%Access control ensures that unauthorized parties are not able to participate in the network, and replay protection is provided by the link layer itself, filtering out frames with sequence numbers smaller than it has already seen.

One of the things the 802.15.4 standard does not specify is how to deal with key establishment and key management, which therefore has to be dealt with in the higher layers. There exists multiple keying models, which essentially are descriptions on what type of key a node uses to communicate with other nodes. Examples of such models include network shared keying, where the entire network share the same symmetric key, and pairwise keying, where each node pair agree upon a key for them to use when communicating with each other. Over the years, many different proposals have surfaced regarding the most secure and efficient way of establishing keys between nodes, and how to perform key management. Some of them have been good, some of them not so good.


\section{6LoWPAN: Putting IP on Top of 802.15.4}

% We want to reach our devices over the Internet. IP is the thing.

Initially, the \gls{ip} was considered to be too "heavy" for low-power wireless networks such as the ones described by the 802.15.4 standard. The idea of implementing \gls{ip} on top of 802.15.4 networks was born as early as in 2001 under the question "Why invent a new protocol when we already have IP?"\cite{Mulligan2007}. With \gls{ip}, the community already had a bundle of existing protocols for management, transport, configuring and debugging, such as \gls{snmp}, \gls{tcp} and \gls{udp}, as well as standardized services for higher layers such as caching, firewalls, load balancing and mobility. Nevertheless, the initial idea of using IP in combination with sensor networks or \gls{wpan}s was not accepted by various groups such as ZigBee \cite{Mulligan2007}. The rejection did, however, not stop the initiative, and a group of engineers within \gls{ietf} started designing and developing what would later be known as \gls{6lowpan}.

A significant advantage with combining \gls{ip} and 802.15.4 is the simplification of the connectivity model between various devices in the networks. As most 802.15.4-based specifications usually needs custom hardware that tend to be complex, the possibilities to interconnect different networks with each other is somewhat limited. By turning to \gls{ip}, the need for complex connectivity models is obviate as it is possible to use well-understood technologies such as bridges and routers. Another advantage with using \gls{ip} is that the routers between the \gls{6lowpan} devices and the outside networks (so-called edge routers) do not need to maintain any state as they are only forwarding datagrams.

\begin{figure}[h]
	\centering
	\includegraphics[scale=0.65]{6lowpan.png}
	\caption{Figure of the \gls{6lowpan} stack, which uses the 802.15.4 physical and link layer, but adds an adoption layer in the network layer.}
	\label{fig:6lowpan-stack}
\end{figure}

\gls{6lowpan} enables wireless 802.15.4 sensor devices to connect directly to the Internet via \gls{ip}v6 by providing an adoption layer in the network layer between it and the data link layer as shown in Figure \ref{fig:6lowpan-stack}. The adoption layer provides an unique functionality which both fragments and compresses incoming packets to enable the embedded devices in 802.15.4 networks to receive the packets while using the least amount of memory and energy \cite{krentz20136lowpan}. Its fundamental idea is that you only "pay" for what you use. \gls{6lowpan} enables a possible total of 320 different header types which can be used in conjunction with each other to specify different types of packets \cite{rfc6282}. This greatly differs from \gls{ip}v4 and Zigbee which defines only one monotonic header, and can be used to greatly minimize the header size of a packet.


The implementation of \gls{6lowpan} proved to not be any more expensive, and in most cases less expensive, compared to other alternatives such as ZigBee or Z-wave in terms of code size and \gls{ram} requirements \cite{Mulligan2007}. \gls{6lowpan} seems to be a natural choice for the future \gls{iot} as a networking protocol. It is scalable thanks to \gls{ip}v6, and can be compressed to a only a few bytes using its fragmentation and compression mechanism. Following the expected bloom in \gls{iot} devices over the next few years (20 billion by 2020), and the fact that the \gls{ip}v6 address space is not going to be exhausted any time soon (roughly $2^{95}$ addresses for each and one of us), \gls{6lowpan} may be a reasonable approach.

However, as neither it or its underlying 802.15.4 layers provide any description on how to perform key establishment and key management, which is and will be of huge importance as the global network of connected \emph{things} grows larger, we need to provide it with safe ways to communicate.


\section{Key Establishment and Key Management}

Todo.

\section{Formal Security Analysis} 


As security protocols grows larger and more complex, they become more and more difficult for humans to analyse. One of the examples of the need for formal security analysis, is the Needham-Schroeder protocol \cite{Needham:1978} from 1978. The Needham-Schroeder protocol is based on public-key cryptography and was intended to allow two communicating parties to mutually authenticate each other.


One of the pioneering work on security analysis was conducted by Burrows, Abadi and Needham with their \gls{ban} logic. \gls{ban} logic is a set of rules which can be used to determine whether received information is legit or not, by formally describe the interaction between communicating parties \cite{burrows1989logic}. It showed promising results in finding security flaws and drawbacks for several authentication protocols, but was later abandoned due to the fact that it verified insecure protocols as secure, and in some cases perfectly sound protocols to be insecure \cite{boyd1993}. One of the protocols that was formally verified using \gls{ban} logic was the Needham-Schroeder protocol.


17 years later, after being deployed and widely used, Lowe discovered, using the automatic tool Casper, that the Needham-Schroeder protocol was in fact insecure, and vulnerable to a man-in-the-middle attack \cite{basin2011model} \cite{lowe1996}. The discovery that such a flaw had gone unnoticed for so many years puzzled the research community, leading to an increased interest in formal security analysis \cite{cremers2009comparing}. Researchers started developing tools for exhaustive search of the problem space of a protocol in order to detect possible abnormalities in protocol behaviour. 

%As the world progressed, adversaries became more sophisticated in terms of what types of operations it can perform, hence resulting in a larger model covering more classes of possible attacks.


%To conduct formal security analysis, we need a formal model to be able to study the protocol under precise assumptions.

The Dolev-Yao model is a formal model used to prove the security properties of cryptographic protocols. While initially being a verification model built for public key protocols, the Dolev-Yao model is also the basis for most of the security analysis done by verification tools \cite{cremers2005operational}. The model is built upon three primary assumptions \cite{dolev1983security}. Firstly, the Dolev-Yao model assumes that the cryptography is perfect, essentially meaning that the cryptographic system can not be tampered with, and an encrypted message can only be decrypted by the party possessing the corresponding decryption key. The second assumption is that the adversary has completely control over the communication network, hence he is able to observe all messages that are sent between communicating parties, and can inject messages given that he is able to forge its content in a valid matter. Lastly, messages that are sent in the network are to be observed as abstract terms, where the attacker has two possible outcomes; either he learns the complete content of the message, or he learns nothing at all. 

More to come.


