\begin{titlingpage}

\noindent
\begin{tabular}{@{}p{4cm}l}
\textbf{Title:} 	& \thetitle \\
\textbf{Student:}	& \theauthor \\
\end{tabular}

\vspace{4ex}
\noindent\textbf{Problem description:}
\vspace{2ex}

\noindent
\gls{iot} is a network where devices, sensors, vehicles, buildings, and humans communicate and collaborate, along with collecting and exchanging information. IEEE 802.15.4 specifies the lower layers for low-rate wireless networks, which are widely seen as the foundation for current IoT communications. One of the potential weaknesses of the IEEE 802.15.4 standard is the lack of specification for key establishment and management.

% What to do
\noindent
This thesis will focus on key management for device-to-device security in \gls{iot}. It will review and compare the proposed protocols, and include both formal and informal security analysis, as well as analysis of both key management requirements and key agreement protocol design for \gls{iot} security. Another goal of the thesis will be to suggest improvements and alternatives to the proposed protocols.

\vspace{6ex}

\noindent
\begin{tabular}{@{}p{4cm}l}
\textbf{Responsible professor:} 	& \theprofessor \\
\textbf{Supervisor:}			& \thesupervisor \\
\end{tabular}

\end{titlingpage}
