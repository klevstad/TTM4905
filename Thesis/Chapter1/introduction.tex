\chapter{Introduction}
\label{chp:introduction} 


\section{Motivation}


Information is the new natural resource. It is around us at all times, the possibilities of reshape it into value are endless, and it is renewable. All we need to do is capture it. Computer devices equipped with sensors are able to capture a certain property of the physical world, and convert it into information. The information can then be exchanged with other devices, processed, computed on, or transformed into something else. To collect as much information as possible, we need a large amount of these devices, and they need to be able to communicate with each other through networks. 

Our information is valuable. Therefore we need to secure the data that we capture to protect it from possible adversaries that would want to intercept, alter, or delete our precious information. Information can be secured by encrypting it, which would make the data look like nonsense to adversaries that intercept it. However, before a device is capable of encrypting its outgoing information stream, it needs to agree upon some key scheme to use for the encryption-decryption process. This is done through key establishment. 

There exists numerous well-tested and deployed protocols for key establishment in wireless networks. These are not, however, always well-suited for device-to-device communication where the devices are meant to be cheap and energy-efficient. Such devices usually computationally strong enough to perform the various operations that nodes in regular wireless networks perform. Therefore, the community needs to rethink their approach when it comes to key establishment schemes for sensor networks.

Protocols for key establishment in device-to-device networks is an emerging market, and involves multiple different network technologies and standards. While in the chase of creating energy efficient and universal key establishment schemes, the security analysis may not always be conducted properly or at all. As the key establishment schemes become more sophisticated and complex, it may become difficult for humans to verify that the scheme is correct and does not contain any states that may cause the protocol to misbehave.

To avoid that insecure protocols are standardized and deployed, which actually have happened, formal security analysis is often conducted to verify that the protocol is in fact correct. Over the recent years, multiple tools for formal security analysis have been developed and made available for the public. By using these tools, it is possible to verify security protocols by allowing a machine to explore each possible state of the protocol in order to expose possible malicious behaviour. Formally security analysis is something that any protocol should be exposed to, but is surprisingly often omitted. Hence, it would be interesting to explore proposed protocols in a formal way to verify that they provide the alleged security stated in the proposal.

\section{Scope and Objectives}

The scope of this thesis is to give an formal security analysis of three key establishment protocols using the tool Scythers. These protocols are utilizing the \gls{ieee} 802.15.4 standard in conjunction with \gls{6lowpan}, which allows for connecting devices to each over the \gls{ip}. In addition to formally verifying the protocols, improvement of the protocols should be suggested. 

* Review three protocols for key establishment in device-to-device communication.

* Formally verify the protocols using a formal security analysis tool.

\section{Methodology}

The first part of this thesis has been a background study of wireless networks such as the \gls{ieee} 802.15.4 and the \gls{6lowpan} to better understand how they provide different types of security services. In addition, different key establishment schemes have been assessed to increase the understanding the properties that are desirable for key establishment schemes in a sensor network setting.

To be able to formally analyse key establishment protocols, a part of the thesis has involved learning how to implement and verify security protocols using the formal security tool known as Scyther. This includes understanding the connection between security properties for key establishment and how Scyther interprets and verifies these properties.

Three protocols for key establishment without any previous formal security analysis have been chosen for this thesis. These are reviewed and discussed to establish their various security properties, as well as security properties that they should possess. The protocols have been modelled using the Scyther tool, and formally verified by. When Scyther verifies a protocol, it returns a table of all the claimed security properties, and an indication whether the property was successfully verified or falsified. In the event of a property being falsified, the attacks have been inspected, and changes to the protocol have been made to achieve the claimed security property. 

\section{Contribution}

The contribution of this thesis is the first published formal security analysis of three key establishment protocols for 802.15.4 networks that utilize \gls{6lowpan}. In addition to formally verify these protocols, this thesis also suggests improvements of the protocols, and explains their applicability for use in real world networks.

% More

\section{Outline}

In Chapter \ref{chp:background}, a general background will be given on the Internet of Things and wireless sensor technology. It will cover the general idea of key establishment, its security properties, and why it is challenging in an \gls{iot} context. In addition, Chapter \ref{chp:background} also contains a brief overview of formal security analysis and the importance of conducting such analysis of modern security protocols.

Chapter \ref{chp:scyther} is an introduction in the formal security analysis tool Scyther, how it works, and what types of security properties that can be formally verified using the tool. In addition to the overall description, the chapter also contains examples of Scyther syntax, how we can model security protocols using the tool, and how to interpret the results of the verification process.

Three suggested protocols for key establishment in sensor networks are introduced in Chapter \ref{chp:protocols}, along with their specifications and weaknesses. These are recently proposed protocols that aim to provide secure key establishment in 802.15.4 networks that utilize \gls{6lowpan}.

Chapter \ref{chp:analysis} describes the formal security analysis of the protocols, how the protocols have been modelled in Scyther, and how the different security properties are assessed. The chapter also contains the results of the verification process with a brief explanation.

The results are discussed and compared in Chapter \ref{chp:discussion}, which also contains suggestions on how to improve the protocols.  In chapter \ref{chp:conclusion}, concluding remarks of the thesis and its contribution are presented. The thesis also comes with an appendix which contains the scripts of the modelled protocols, as well as a overview of the notations that are used when describing the protocols in detail.