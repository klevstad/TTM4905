\chapter{Introduction}
\label{chp:introduction} 

This should be the introduction to the thesis.


\section{Motivation}

In today's world, more and more data are exchanged between users, devices, and the Internet. We like to have our data accessible regardless of device, and we need to be mobile. This calls for device-to-device communication, preferably through the Internet or other local networks.



\section{Scope}

The scope of this thesis is to give an formal security analysis of key establishment protocols using the tool Scyther, and formally verify three proposed protocols. These protocols are utilizing the \gls{ieee} 802.15.4 standard in conjunction with \gls{6lowpan}, which allows for connecting devices to each over the \gls{ip}. 

\section{Method}

Scyther

\section{Outline}

In Chapter \ref{chp:background}, a general background will be given on the Internet of Things and wireless sensor technology. It will also cover the general idea of key establishment, its importance, and why it is challenging in an \gls{iot} context, and also a brief overview of formal security analysis. Chapter \ref{chp:scyther} contains an introduction in the formal security analysis tool Scyther, along with examples on how to model protocols using the tool. Three suggested protocols for key establishment in sensor networks are introduced in Chapter \ref{chp:protocols}, before they are formally verified and the results are presented in Chapter \ref{chp:analysis}. The results are discussed and compared in Chapter \ref{chp:discussion}, before the conclusion is presented in Chapter \ref{chp:conclusion}. The thesis also comes with an appendix which contains the scripts of the modelled protocols, as well as a overview of the notations that are used when describing the protocols in detail.