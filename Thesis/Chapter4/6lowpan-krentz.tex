\chapter{6LoWPAN Security - Adding Compromise Resilience to the 802.15.4 Security Sublayer}
\label{chp:krentz-6lowpan}


\gls{apkes} is a proposed protocol for handling key establishment and key management in \gls{6lowpan}. It is currently implemented in the operating system Contiki, which is an operating system targeted at the sensor network community, but has not undergone any formal security analysis. This chapter will cover its ideas, and conduct a formal security analysis using Scyther. 

\section{Introduction to APKES}

As previously described, \gls{6lowpan} is a protocol stack for integrating \gls{wsn}s running on 802.15.4 with \gls{ip}v6 networks, and enables the nodes in the network to communicate with each other or remote hosts over \gls{ip}. \gls{apkes} provides a framework for establishing pairwise keys for nodes in \gls{6lowpan} networks. The advantage with pairwise keys over other key schemes such as a network-shared key is related to node compromises. In \gls{6lowpan} networks, devices are often placed in potential hostile and unattended areas, greatly increasing the possibility of being tampered with by attackers. In the case of a network shared key, the whole network would be compromised in the event of a node compromise. Also, the attacker would be able to add new nodes to the network, as the upper-layer protocols rely on the 802.15.4 security sub-layer which is able to filter out replayed packets and prevent injection, but not discover node compromises \cite{krentz20136lowpan}. A solution to the tampering problem could be to construct tampering-proof nodes, but this is expensive and difficult, hence not a preferable solution \cite{anderson1996tamper}. Pairwise keys, however, would only compromise the communication going to or from that particular node. Figure \ref{fig:6lowpan-krentz} illustrates how \gls{apkes} is implemented at the link layer along with the 820.15.4 security sublayer.

\begin{figure}
	\centering
	\includegraphics[scale=0.80]{6lowpan-krentz.png}
	\caption{\gls{apkes} is positioned in the data link layer in the \gls{6lowpan} stack expanding the 802.15.4 security sublayer \cite{krentz20136lowpan}.}
	\label{fig:6lowpan-krentz}
\end{figure}

\gls{apkes} provides a ``pluggable'' key establishment scheme for \gls{6lowpan} networks using pairwise keys, where the developer of a \gls{6lowpan} network picks an appropriate key establishment scheme and delegates \gls{apkes} into handling the key establishment with other nodes \cite{krentz20136lowpan}. As there is really no superior scheme for \gls{6lowpan} networks, the use of pluggable schemes enhances the overall usability of the protocol, as the developer can use the most appropriate scheme based on the challenges he faces. The only function of the plugged-in scheme is to feed \gls{apkes} with the shared secret for communicating nodes, and \gls{apkes} will handle both key establishment and key management. Examples of pluggable schemes that has been suggested for \gls{apkes} are \gls{leap} \cite{zhu2006leap+}, Blom's Scheme \cite{blom1984optimal}, and random pairwise keys \cite{chan2003random}. In the case of random pairwise keys, path key establishment has to be implemented in addition to \gls{apkes}.

Authentication (and if enabled, encryption) of unicast frames, which are frames sent from one node to another, is achieved through \gls{apkes}. \gls{mic}s are generated by the 802.15.4 security sublayer, through the use of \gls{ccm}*, which is a modified version of the regular \gls{ccm} allowing for the payload of the frame to be encrypted using \gls{aes} with a 128-bit key \cite{krentz20136lowpan}.

% Nevne at LEAP er best/mest snill mot strøm og minne?



\subsection{Easy Broadcast Encryption and Authentication Protocol}

Authentication of multicast frames, however, which are sent from one node to many others and used for node discovering and network changes, are not authenticated by \gls{apkes}. Broadcast frames are authenticated with keys that are shared between neighbours, meaning that an attacker that compromises a node would not only gain access to its broadcast key, but also its neighbours' broadcast keys. As a solution for authenticating broadcast keys to achieve compromise-resilience, Krentz, Rafiee, and Meinel has suggested \gls{ebeap} to go along with \gls{apkes}. 


\subsection{Protocol specification}

\begin{figure}[h]
\begin{tcolorbox}[title=Three-way handshake in APKES]
\begin{align*}
U:\ & Generate\ R_u\ randomly\\
U \rightarrow *:\ & \texttt{HELLO}\langle{R_u}\rangle{}\\
V:\ & Generate\ R_v\ randomly.\ Wait\ for\ T_w \leq M_w\\
V:\ & K_{v,u}\ from\ pluggable\ scheme\\
V \rightarrow U:\ & \texttt{HELLOACK}\langle{R_u, R_v}\rangle{K_{v,u}}\\
V:\ & K'_{v,u}\ =\ AES(K_{v,u}, R_u || R_v)\\
U:\ & K_{u,v}\ from\ pluggable\ scheme\\
U:\ & K'_{u,v}\ =\ AES(K_{u,v}, R_u || R_v)\\
U \rightarrow V:\ & \texttt{ACK}\langle{}\rangle{K'_{u,v}}
\end{align*}
\end{tcolorbox}
\caption{Figure of the messages sent between communicating parties during \gls{apkes}' three-way handshake. $\langle{msg}\rangle{K}$ indicates that the frame is authenticated using the key $K$.}
\label{fig:apkes-handshake}
\end{figure}


\section{Formal Security Analysis of APKES}


\subsection{Key establishment}

\subsection{Key agreement}

\subsection{Security properties}

\subsubsection{Authentication}

Key authentication = Secrecy of keys

Entity authentication = aliveness


implicit authentication -> Do not know whether or not the other party possesses the session key (yet). Explicit --> got a MAC signed using the session key


HELLOACK: Implicit key authentication
ACK: Explicit key authentication ????

\subsubsection{Key Compromise Impersonation}


% kok
The LKR actor rule allows the adversary to learn the long-term key of the agent executing the test thread (also called the actor). The intuition is that a protocol may still function as long as the long-term keys of the other partners are not revealed. This rule allows the adversary to perform so-called Key Compromise Impersonation attacks [21]. The rule's second premise is required because our model allows agents to communicate with themselves.

\subsubsection{Unknown Key Share}

\subsubsection{key confirmation}

non-injective synch / non-injective agreement

\subsubsection{Forward Secrecy}

Mutual forward secrecy
weak perfect forward secrecy
perfect forward secrecy


\subsection{Threat model}

\subsection{Formal model}

\subsubsection{Specifying security properties}

\subsection{Results}

\subsection{Recommendations}




%leap master key = "ephemeral key"


%
%\begin{figure}
%	\centering
%	\includegraphics[scale=0.55]{apkes.png}
%	\caption{\gls{apkes} consists of a three-way handshake, and utilizes a pluggable scheme for deriving the shared secret between two nodes \cite{krentz20136lowpan}. $\langle{msg}\rangle{K}$ indicates that the frame is authenticated using the key $K$.}
%	\label{fig:apkes}
%\end{figure}



%\chapter{TBA}

%\chapter{Conclusion}


%\chapter{Appendix}