\chapter{Conclusion}
\label{chp:conclusion}


In this thesis, the first formal security analyses of three proposed protocols for key establishment in \gls{ieee} 802.15.4 networks that utilize \gls{6lowpan} are presented. The protocols have been investigated, reviewed, and formally verified using the tool Scyther. Scyther, the selected tool for verifying these protocols, has also been examined and explained in details to aid the reader in understanding the importance of verifying the correctness and security of security protocols, and how this can be done using computer software. Key establishment, different schemes, and the desirable properties in key establishment have also been thoroughly assessed and explained to support the analysis and to explain the modelling choices.

There exist multiple architectures for key establishment schemes, namely those based on symmetric and asymmetric encryption, as well as those that leverage online key servers and trusted third parties. This thesis has identified some of the reasons for choosing a symmetric key establishment scheme over key servers and public-key cryptography but has also discussed the possibility of using a hybrid system where the infrastructure allows for having more powerful devices in the \gls{6lowpan} network to handle the heavier computation. However, as technological progress often leads to smaller devices and new business opportunities, symmetric key establishment schemes are suited for future applications because of their low-complexity and low energy consumption.


Based on the results that been presented, \gls{akes} seems to be a valid and usable scheme for establishing keys in a \gls{6lowpan} network. Both \gls{apkes} and \gls{akes} have been formally verified for an unbounded state space, and can be proven to be correct schemes that may have an appropriate role in a real-life network. \gls{akes}, however, holds advantages over \gls{apkes} when it comes to providing mobility in networks, which can be assumed to be a requirement for modern device-to-device communication in dynamic networks.

Flaws have been presented and explained in the \gls{sakes} protocol, along with suggested changes that can improve the protocol. However, due to the design of \gls{sakes}, it has been infeasible to provide an analysis of the protocol in a single model. Therefore, the protocol has been divided into two separate models targeting the authentication and key establishment phases. Because of this separation, the security analysis is not entirely complete, and there may exist attacks that went undiscovered through the formal analysis presented in this thesis. \gls{sakes} provides an interesting authentication phase, which holds advantages over the two other protocols, given that the authentication module used in the scheme is trusted and secure.

\section{Future work}

As the formal security analysis of \gls{sakes} was separated into two new models, and formally verified individually, it raises the question whether it exists undiscovered attacks on the protocol. One way to verify this unanswered question is to use a more powerful computer to search through the state-space of the protocol, for example, \gls{ntnu}'s super-computer Vilje.  

Also, as this thesis presents multiple fixes that may improve the usability of \gls{sakes} as a key establishment protocol in \gls{6lowpan} networks, an compelling case for future work would be to implement the protocol and analyse it in a testbed environment to verify its applicability in a real-life network. This work would include more extensive security analysis and also an analysis of the energy consumption of the different entities in \gls{sakes} to verify whether standard technologies can run the protocol in an efficient matter.

% Intro

% Resultater



% Diskusjon

%